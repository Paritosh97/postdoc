% This is samplepaper.tex, a sample chapter demonstrating the
% LLNCS macro package for Springer Computer Science proceedings;
% Version 2.21 of 2022/01/12
%
\documentclass[runningheads]{llncs}
%
\usepackage[T1]{fontenc}
% T1 fonts will be used to generate the final print and online PDFs,
% so please use T1 fonts in your manuscript whenever possible.
% Other font encondings may result in incorrect characters.
%
\usepackage{graphicx}
% Used for displaying a sample figure. If possible, figure files should
% be included in EPS format.
%
% If you use the hyperref package, please uncomment the following two lines
% to display URLs in blue roman font according to Springer's eBook style:
%\usepackage{color}
%\renewcommand\UrlFont{\color{blue}\rmfamily}
%
\begin{document}
%
\title{Contribution Title\thanks{Supported by organization x.}}
%
%\titlerunning{Abbreviated paper title}
% If the paper title is too long for the running head, you can set
% an abbreviated paper title here
%
\author{First Author\inst{1}\orcidID{0000-1111-2222-3333} \and
Second Author\inst{2,3}\orcidID{1111-2222-3333-4444} \and
Third Author\inst{3}\orcidID{2222--3333-4444-5555}}
%
\authorrunning{F. Author et al.}
% First names are abbreviated in the running head.
% If there are more than two authors, 'et al.' is used.
%
\institute{Princeton University, Princeton NJ 08544, USA \and
Springer Heidelberg, Tiergartenstr. 17, 69121 Heidelberg, Germany
\email{lncs@springer.com}\\
\url{http://www.springer.com/gp/computer-science/lncs} \and
ABC Institute, Rupert-Karls-University Heidelberg, Heidelberg, Germany\\
\email{\{abc,lncs\}@uni-heidelberg.de}}
%
\maketitle              % typeset the header of the contribution
%
\begin{abstract}
The abstract should briefly summarize the contents of the paper in
150--250 words.

\keywords{First keyword  \and Second keyword \and Another keyword.}
\end{abstract}
%
%
%

\section{Introduction}


\section{Related Work}

\subsection{Pedestrian Navigation and Traffic Safety Tasks}

\subsection{Scenario Description Languages and Task Specification}

\subsection{Human-in-the-Loop Simulation and Virtual Reality}

\subsection{Large Language Models for Structured Scenario Generation}



\section{Method}


\subsection{Pedestrian Navigation Task Taxonomy}
\label{sec:task_taxonomy}

\subsubsection{Scope and Inclusion Criteria}

This work targets the generation of pedestrian navigation scenarios for
human-controlled users in XR-based simulation.
The pedestrian itself is not behaviorally simulated; instead, the
scenario specifies navigation objectives, constraints, triggers, and
evaluation conditions, while execution and decision-making are performed
by the user.

We define a pedestrian navigation task as an atomic interaction unit
that satisfies the following criteria:

\begin{itemize}
    \item \textbf{Intentionality}: the task encodes a clear navigation
    objective or decision the user must perform.
    \item \textbf{Environmental grounding}: the task is defined with
    respect to explicit spatial, traffic, or semantic elements of the
    environment.
    \item \textbf{Evaluability}: the task admits objective success and
    failure conditions observable by the simulator.
    \item \textbf{Composability}: the task can be combined sequentially
    or hierarchically with other tasks to form longer scenarios.
    \item \textbf{Scenario expressibility}: the task can be represented
    declaratively using OpenSCENARIO-style constructs (e.g., goals,
    constraints, triggers).
\end{itemize}

Tasks that differ only in low-level motion execution are excluded, as
locomotion is performed by the user and not modeled by the system.

\subsubsection{Task Taxonomy}

Table~\ref{tab:ped_tasks} summarizes the resulting pedestrian navigation
task taxonomy, including task definitions and corresponding success and
failure conditions.

\begin{table}[t]
\centering
\footnotesize
\setlength{\tabcolsep}{4pt}
\renewcommand{\arraystretch}{1.15}
\begin{tabular}{p{2.4cm} p{3.1cm} p{3.2cm} p{2.6cm}}
\hline
\textbf{Task} &
\textbf{Description} &
\textbf{Example Scenario} &
\textbf{Success / Failure} \\
\hline

Point-to-point &
Reach a target location via walkable areas &
Walk from building entrance to bus stop &
Arrive within tolerance / enter restricted area \\

Crosswalk-based &
Cross a road using a designated crosswalk &
Cross a two-lane street at a marked crosswalk &
Stay within crosswalk / leave crosswalk \\

Multi-step routes &
Complete ordered sequence of sub-goals &
Sidewalk $\rightarrow$ crosswalk $\rightarrow$ bus stop &
All steps in order / skipped or reordered step \\

Traffic-aware crossing &
Cross only when a safe traffic gap exists &
Wait for vehicles to pass before crossing &
Gap above threshold / unsafe gap \\

Constrained path navigation &
Navigate using permitted paths only &
Follow sidewalk around a fenced area &
Compliant arrival / forbidden shortcut \\

Environmental constrained &
Avoid static or semantic obstacles &
Navigate around construction barriers &
No restricted-zone entry / collision or violation \\

Triggered navigation &
Begin movement after an event trigger &
Start walking when pedestrian light turns green &
Move after trigger / premature movement \\

Tolerance-based navigation &
Reach target with positional precision &
Stand at a bus stop marker &
Within tolerance region / outside bounds \\

Dynamic navigation &
Adapt path due to dynamic obstacles &
Re-route when another pedestrian blocks the path &
Successful adaptation / deadlock \\

Speed adaptation &
Adapt walking speed to context &
Slow down because of a slower entity in front &
Speed within limits / speed violation \\

\hline
\end{tabular}
\caption{Pedestrian navigation task taxonomy with representative example scenarios and evaluation criteria.}
\label{tab:ped_tasks}
\end{table}


\section{Implementation}

\section{Evaluation}
\section{Conclusion}

\subsubsection{Acknowledgements} 

\bibliographystyle{splncs04}
\bibliography{main}
\end{document}
